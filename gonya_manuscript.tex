\documentclass[12pt]{article}
 \usepackage[hcentering,bindingoffset=20mm]{geometry}
 \usepackage{placeins}
 \usepackage[numbib]{tocbibind}
 \usepackage{rotating}
\usepackage[square,sort,comma,numbers]{natbib}
 \usepackage{graphicx}
 \usepackage{tabularx}
 \linespread{1.3}
 \usepackage{gensymb}
 \usepackage{longtable}
 \usepackage{lscape}
 \usepackage{url}
 \addtolength{\textwidth}{2cm}
 \addtolength{\hoffset}{-1cm}
 
 
 \addtolength{\textheight}{2cm}
 \addtolength{\voffset}{-1cm}
 \setlength{\parindent}{0pt}
 
\title{Gonyalacales phylogeny. $^{1}$}
\author{me}
\date{}

\begin{document}
\maketitle
\paragraph{}Anna Liza Kretzschmar$^{2}$\\
Climate Change Cluster (C3), University of Technology Sydney, Ultimo, 2007 NSW, Australia, anna.kretzschmar@uts.edu.au
\paragraph{}Aaron Darling \\
Ithree institute (i3), University of Technology Sydney, Ultimo, 2007 NSW, Australia
\paragraph{}Shauna Murray\\ 
Climate Change Cluster (C3), University of Technology Sydney, Ultimo, 2007 NSW, Australia
\newpage
\section{Abstract}


\FloatBarrier
\begin{table}
\caption{Transcriptomes used for study along with taxonomic placement at family level and source. Family level placement derived from algaebase. MMETSP abbreviation for marine Microbial eukaryotic transcriptome sequencing project, by Moore Foundation.}
\label{tbl:Transcriptomes}
\begin{tabular}{  | p{3cm} |p{4cm} | p{2cm} | p{3cm} | p{2cm} | p{2cm} | p{2cm} |}
\hline
\textbf{Family}&\textbf{Species}&\textbf{Strain}&\textbf{ID}&\textbf{CEGMA score}&\textbf{BUSCO single complete}&\textbf{Reference}\\
\hline
 \multicolumn{7}{| c |}{Gonyaulacales transcriptomes}\\
    \hline
   Ceratiaceae&\emph{Ceratium fusus}&PA161109 &MMETSP1074&79.44&202&\citep{keeling2014marine}\\
        \hline
  Crypthecodiniaceae&\emph{Crypthecodinium cohnii}&Seligo&MMETSP0326\_2&83.87&202&\citep{keeling2014marine}\\
        \hline
    &\emph{Alexandrium catanella}&OF101&MMETSP0790&62.10&178&\citep{keeling2014marine}\\
        \hline
    &\emph{Alexandrium monilatum}&JR08&MMETSP0093&77.82&208&\citep{keeling2014marine}\\
        \hline
&\emph{Pyrodinium bahamense}&pbaha01&MMETSP0796&79.84&197&\citep{keeling2014marine}\\
        \hline
Gonyaulacaceae&\emph{Gambierdiscus australes}&CAWD149&MMETSP0766\_2&73.79&172&\citep{keeling2014marine}\\
        \hline
    &\emph{Gambierdiscus caribaeus}&&Price et al.&84.27&104&\citep{price2016analysis}\\
        \hline
    &\emph{Gambierdiscus lapillus}&HG4&This study&85.08&227& \\
        \hline
    &\emph{Gambierdiscus} cf. \emph{silvae}&HG5&This study&77.42&215& \\
        \hline
    &\emph{Gonyaulax spinifera}&CCMP409&MMETSP1439&67.34&131&\citep{keeling2014marine}\\
        \hline
    &\emph{Lingulodinium polyedra}&CCMP1738&MMETSP1032&77.82&209&\citep{keeling2014marine}\\
        \hline
Protoceratiaceae&\emph{Protoceratium reticulatum}&CCCM535=CCMP1889&MMETSP0228&61.69&176&\citep{keeling2014marine}\\
    \hline
    \multicolumn{7}{| c |}{Peridiniales transcriptomes}\\
 \hline
 \multicolumn{7}{| c |}{Outgroup transcriptomes}\\
 \hline
     \hline
Dinophysiaceae&\emph{Dinophysis acimunata}&DAEP01&MMETSP0797&63.71&178&\citep{keeling2014marine}\\
        \hline
Dinophyceae incertae sedis&\emph{Azadinium spinosum}&3D9&MMETSP1036\_2&80.65&193&\citep{keeling2014marine}\\
        \hline
Gymnodiniales&\emph{Karenia brevis}&CCMP2229&MMETSP0030&80.65&184&\citep{keeling2014marine}\\
    \hline
\end{tabular}
\end{table}
Azadinium Order Dinophyceae incerta sedis according to algaebase, gonyaulacales from NCBI taxonomy
\newpage
\bibliographystyle{acm}
\bibliography{gonya.bib}


\end{document}